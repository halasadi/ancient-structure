\documentclass[11pt, DIV10,a4paper]{article}

\usepackage{color}
\usepackage{float}
\usepackage[bf,small]{caption2}
\usepackage{comment}
\usepackage{amsmath}
\usepackage{bigints}
%\usepackage{caption}
\usepackage{rotating, booktabs}
%\usepackage[retainorgcmds]{IEEEtrantools}
\usepackage{amsopn}
\usepackage{amsfonts}
\usepackage{amsthm}
\usepackage{amssymb}
\usepackage{amsbsy}
\usepackage{multirow}
%\usepackage{slashbox}
%\usepackage{titling}
%\usepackage[numbers]{natbib}
\usepackage{natbib}
\usepackage{tocbibind}
\hyphenpenalty=1000
\usepackage[a4paper,colorlinks,breaklinks,bookmarksopen,bookmarksnumbered]{hyperref}
\usepackage[refpage]{nomencl}
\usepackage{makeidx}
\usepackage{pst-all}
\usepackage{epsfig,psfrag}
\usepackage{graphicx}
%\usepackage{undertilde}
%\usepackage{breqn}
\usepackage{amsmath}
\usepackage{epstopdf}
\usepackage{tabularx}
%\usepackage{mathtools}
%\usepackage{algorithm}
%\usepackage{algorithmic}
\usepackage{subfigure}
\usepackage{setspace}
\usepackage[mathscr]{euscript}
\usepackage[margin=1in]{geometry}
\singlespacing
\newcommand{\bzero}{\boldsymbol{0}}
\newcommand{\bone}{\boldsymbol{1}}
\newcommand{\bmu}{\boldsymbol{\mu}}
\newcommand{\bSigma}{\boldsymbol{\Sigma}}
\newcommand{\bI}{\bf{I}}
\newcommand{\bv}{\bold{v}}

\newcommand {\ctn}{\cite}

\pagenumbering{arabic}


\usepackage{url}
\hyphenpenalty=1000
%\documentclass[]{gSCS2e}


\newtheorem{lemma}{Lemma}[section]
\newtheorem{proposition}{Proposition}[section]
\newtheorem{corollary}{Corollary}[section]
\newtheorem{theorem}{Theorem}[section]
\newtheorem{example}{Example}[section]
\newtheorem{result}{Result}[section]
\normalsize
\setlength\parindent{0pt}

\begin{document}

\section{EM model incorporating sample size error}

The E-step log likelihood under the EM algorithm has the following form 

$$ E_{Z,M} | X, f^{(n)}_{k}, Q^{(n)}, F^{(n)}_{u}, F^{(n)}_{k} \left [ log P (X,Z,M, f^{(n)}_{k} | Q^{(n)}, F^{(n)}_{u}, F^{(n)}_{k} )\right ] $$

We first determine the conditional distribution of $Z$ and $M$ given X, Q and F as in Skotte et al 2013. 

$$ P(Z_{ij}=t, M_{ij}=1 | X_{ij}, f^{(n)}_{k}, Q^{(n)}, F^{(n)}_{u}, F^{(n)}_{k} ) = c^{(ijt)}_{n} X_{ij} /2  $$

$$ P(Z_{ij}=t, M_{ij}=0 | X_{ij}, f^{(n)}_{k}, Q^{(n)}, F^{(n)}_{u}, F^{(n)}_{k} ) = d^{(ijt)}_{n} (2 - X_{ij}) /2  $$

\vspace{0.2 in}

When the updates do not distinguish between $F_{k}$ and $F_{u}$, we will use the notation $F$ for the sake of readability.

\vspace{0.2 in}

$$ c^{(ijt)}_{n} = \frac{Q^{n}_{it} F^{(n)}_{j,k}} {\sum_{l} Q^{n}_{il} F^{(n)}_{j,l}} $$

$$ d^{(ijt)}_{n} = \frac{Q^{n}_{it} (1 - F^{(n)}_{j,k})} {\sum_{l} Q^{n}_{il} (1 - F^{(n)}_{j,l})} $$

 In addition,
 
 $$ P(Z_{ij}=t, M_{ij}=1 | Q^{(n)},F^{(n)}) = Q^{n}_{it} F^{(n)}_{j,t} $$
 
 $$ P(Z_{ij}=t, M_{ij}=0 | Q^{(n)},F^{(n)}) = Q^{n}_{it} (1 - F^{(n)}_{j,t}) $$
  
  
We can write down the E step log likelihood in that case as 

\begin{align*}
& E_{Z,M} | X, f^{(n)}_{k}, Q^{(n)}, F^{(n)}_{u}, F^{(n)}_{k} \left [ log P (X,Z,M, f_{k} | Q, F_{u}, F_{k} )\right ] &  \\
& =  \qquad E_{Z,M} | X, f^{(n)}_{k}, Q^{(n)}, F^{(n)}_{u}, F^{(n)}_{k} \left [ log P (X,Z,M | Q, F_{u}, F_{k} ) + log P(f_{k}| F_{k}) \right ] \\
&= \qquad \sum_{i,j} E_{Z_{ij},M_{ij} | X_{ij},f^{(n)}_{k}, Q^{(n)}, F^{(n)}_{u}, F^{(n)}_{k}}  \left [ log P (X_{ij},Z_{ij},M_{ij} | Q,  F_{u}, F_{k} ) \right]  + log P (f_{k} | F_{k}) \\
& \propto \qquad  \sum_{ij} \sum_{t=1}^{T_u}  \left [ log (q_{it}F^{jt}_{u}) a^{ijt}_{n} + log(q_{it}(1 - F^{(jt)}_{u})) b^{ijt}_{n} \right ]+  \sum_{ij} \sum_{t=1}^{T_k}  \left [ log (q_{it}F^{jt}_{k}) a^{ijt}_{n} + log(q_{it}(1 - F^{(jt)}_{k})) b^{ijt}_{n}  \right ] + \\
& \qquad \qquad \sum_{j}\sum_{t=1}^{T_{k}}  \left [ f_{jt}n_{t} log(F_{jt}) + n_{t} (1 - f_{jt}) log(1 - F_{jt}) \right ] \\
\end{align*}

here 

$$ a^{ijt}_{n} = c^{ijt}_{n} X_{ij}/2 $$

$$ b^{ijt}_{n} = d^{ijt}_{n} (2 - X_{ij})/2 $$

Take the derivative with respect to $F_u$ and solve for roots  to get the $(n+1)$ th update of $F_u$ to be 

$$ F^{jt}_{u,(n+1)} =  \frac{\sum_i a^{ijt}_n}{\sum_i a^{ijt}_n + \sum_i b^{ijt}_n} $$


Take the derivative with respect to $F_k$ and solve for roots  to get the $(n+1)$ th update of $F_u$ to be 

\begin{align*}
F^{jt}_{k,(n+1)} & =  \frac{\sum_i a^{ijt}_n + n_t  f_{jt}}{\sum_i a^{ijt}_n + \sum_i b^{ijt}_n + n_t} \\
& =   \left ( \frac{\sum_i a^{ijt}_n + \sum_i b^{ijt}_n}{\sum_i a^{ijt}_n + \sum_i b^{ijt}_n+n_t} \right )\frac{\sum_i a^{ijt}_n}{\sum_i a^{ijt}_n + \sum_i b^{ijt}_n} + \left ( \frac{n_t}{ \sum_i a^{ijt}_n + \sum_i b^{ijt}_n+n_t} \right ) f_{jt} 
\end{align*}

Take the derivative with respect to $Q$ with constraints and solve for roots  to get the $(n+1)$ th update of $Q$ to be 

$$ q^{(n+1)}_{it} = \frac{1}{M} \sum_{j=1}^{M} \left ( a^{ijt}_{n} + b^{ijt}_{n} \right ) $$








\end{document}


